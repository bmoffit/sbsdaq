\documentclass{chowto}
%\documentstyle[12pt,epsfig]{article}
%\textwidth=155mm
%\textheight=235mm
%\setlength{\voffset}{-8.0mm}
%\oddsidemargin 5.mm
%\evensidemargin 5.mm
 \title{Electromagnetic calorimeter BigCal}
\howtotype{user}
\experiment{GEp(III) and two-gamma}
\author{C.F. Perdrisat}
\category{general}
\maintainer{C.F. Perdrisat}
\date{9/23/2007}

\input epsf

\begin{document}

\begin{abstract}
A description of the physical configuration of the calorimeter built for 
the GEp(III) experiment, followed by one of the electronic layout leading 
to the generation of the calorimeter trigger.
\end{abstract}
\section{General Description}

The electromagnetic calorimeter BigCal consists of 1024 bars of lead glass, 
3.8x3.8 cm$^2$ cross section and 45 cm length, stacked in a 32 by 32 array, 
and 720 bars 4.0x4.0$cm^2$ 40 cm long, stacked 30 horizontal time 24 
vertical, on top of the first 1024. The glass is TF0-1, density 
3.86 $gcm^{-3}$, index of refraction $n=$1.6522 and radiation length 2.74 
$gcm^{-2}$. The nuclear absorption length is 22 $cm$. 
The arrangement of the glass bars is seen in Fig. \ref{fig:glassback}.
In the two 2007-8 GEp experiments, the main functions of BigCal are to 
help separate the electron of elastic $ep$ scattering from the background, 
by 1) providing a trigger signal, and 2) determining the $x$ and $y$ 
position of the shower associated with the elastic electron; the expected 
position resolution is 
in the range 0.5 to 1.0 cm, depending on electron energy and angle 
relative to the normal to the calorimeter face. The corresponding electron 
angles will then be compared with the expected values calculated from the
angles (and momentum) of the proton detected in the HMS. The electron energy 
is an additional, secondary information obtained by summing the signal in 
the bars 
containing the electron shower; it can be calculated in software with 
appropriate cuts on time and pulse height, and more
coarsely directly from the summing hardware used to 
define the trigger. 
\begin{figure}[h]
%\begin{minipage}[b]{0.45\linewidth}
\begin{center}
\epsfig{file=/home/perdrisa/PS/calo_logic_5.eps,height=4.0in}
\caption[]{\it{The 1744 bars of lead glass in BigCal.The colors are used to 
distinguish subgroups of 64 bars as defined by the second level trigger.}}
\label{fig:glassback}
\end{center}
\end{figure}
A trigger threshold will be set empirically to  
keep the trigger rate low enough;it will be typically higher than 50\% 
of the elastic electron energy.The expected energy resolution is about 5\% 
for 1 GeV electrons.

\subsection{Mechanicals}
\label{mechanics}
\subsubsection{Physical description}
\begin{figure}[h]
%\begin{minipage}[b]{0.45\linewidth}
\begin{center}
\epsfig{file=/home/perdrisa/PS/calo_box_backview_rev.eps,height=4.0in}
\caption[]{\it{A longitudinal cut through the calorimeter on the right, 
and a rear
view showing details of the lead glass stack, PM and PM bases, as well 
as the patch panels attached on vertical bars in the back of the setup.}}
\label{fig:schematicview}
\end{center}
\end{figure}

The lead glass stack is contained in a frame attached to the calorimeter 
platform. All 56 layers are resting against the right internal wall of the 
frame (seen from behind the calorimeter). The bars have all been measured, and chosen so as to minimize 
the fluctuation in height, from layer to layer; they are kept in place 
by horizontal pushers. The total weight of the glass alone is 4300 kg. 

The lead glass bars in the lower part of the calorimeter are arranged in 32
rows of 32 bars horizontally, for a total of 1024 bars; the glass originated 
in the HEP group in Protvino, Russia. The upper part of 
the calorimeter consists of 24 rows of 30 bars horizontally, for a total of 
720 bars; this glass was used in the Hall A RCS experiment, but is originally 
from Yerevan. A schematic of the various components of the calorimeter 
proper is in 
Fig.~\ref{fig:schematicview}. Each bar is connected to a 12-stage, venetian
blind Russian FEU-84 
photomultipliers (PM) 
through a 5 mm thick Si-pad (cookies), and held in place in 
horizontal 2'' thick aluminum cross bars, one for 4 rows of PM's; each 
cross bar holds 
128 PMs in the lower section, and 120 in the upper section. There are 14 
cross bars in total; they are attached to the frame and their position is
adjustable. The base of each PM is attached to the cross bar by two threaded 
rods screwed into the cross bar, applying pressure on the cookies. 

The numbering of the bars is shown in Fig. \ref{fig:glassback}: looking at the 
calorimeter from the back, the first bar at the bottom left corner is bar 1.
Although the upper section (RCS) has  only 30 bars per row, as compared to the 
lower part's 32 bar per row, the horizontal size of the two sections of glass 
differ by only 1.6 cm. Electronically the calorimeter is divided into 
for columns A, B, C and D, 
each of 8 bars; accordingly in the RCS section bars 16 and 32 are physically 
absent. 

\subsubsection{Photomultiplier High Voltage}

The photomultiplier voltage dividers of the Protvino section of 1024 bars
have the voltage of the 4 last stages provided by 4 external power supplies.
The resistive part of the voltage divider chain has a resistance of 5.6 
M$\Omega$. The last for dynodes are coupled by 1.5 nF capacitors. The load
of the 4 external power supplies consists of 4 variable shunt resistors of 20
k$\Omega$ each; it needs to be adjusted when the number of PM's connected is 
changed. A schematic of the bases power supplies and load resistors is in 
Fig.~\ref{fig:PMHV}. The power dissipated per tube, with -1300 V applied to the cathode, and -300V distributed over the last 4 dynodes, is 0.18 W.
The same photomultipliers are used in the RCS section; there the base is of 
standard design, with a total resistance of 1.7 M$\Omega$; with an applied 
voltage of -1300 V the power dissipated is 1 W per base. The total heat 
dissipation in the black-box is then 
1034$\times$0.18+720$\times$1=185 + 720=900 W. The black-box can be 
ventilated by two external turbofans.  

\begin{figure}[h]
%\begin{minipage}[b]{0.45\linewidth}
\begin{center}

\epsfig{file=/home/perdrisa/PS/LP_Base_and_booster.eps,height=3.0in}
\caption[]{\it{Wiring of the Protvino PM bases with the 4 booster voltage supplies and corresponding load resistors.}}
\label{fig:PMHV}
\end{center}
\end{figure}

\subsection{Electronics}
\label{electronics}
\subsubsection{First level summing}

\begin{figure}[h]
%\begin{minipage}[b]{0.45\linewidth}
\begin{center}
%\epsfig{file=/home/perdrisa/PS/multiplex_firstlevel_rev.eps,height=5.0in,angle%=-90}
\epsfig{file=/home/perdrisa/PS/multiplex_firstlevel_rev.eps,height=7.4in}
\caption[]{\it{}}
\label{fig:firstlevel}
\end{center}
\end{figure}

\small{
\begin{table}[h]
\caption[]{\it Content of the 14 NIM bins housing the 128 ``adder'' modules, 
each with two input octets. This is to be correlated with Fig.~ 
\ref{fig:firstlevel} which shows
only the first pair of NIM bins (left and right) for the Protvino (PROT) and 
RCS/Yerevan (RCS) parts of BigCal.}
\begin{center}
\begin{tabular}{|c|c|c|c|c|c|c|c|c|c|c|c|}
\hline
\it NIM & & & & & & \it NIM & & & & &  \\ \hline
\it bin&\it Left  & & & & & \it bin & \it Right & & & & \\ \hline
 & \it slot & \it 1,2 &\it 3,4 &\it 8,9 & \it 10,11 & & \it slot&\it 1,2 &\it 3,4 &
\it 8,9 &\it 10,11 \\ \hline
\it 1& &6 & 5 & 2 & 1 & & \it 5 & 3 & 4 & 7 & 8 \\

\it 2 & &14 & 13 & 10 & 9 & & \it 6 & 11 & 12 & 15 & 16 \\

\it 3 & &22 & 21 & 18 & 17 & & \it 7 & 19 & 20 & 23 & 24 \\

\it 4 & &30 & 29 & 26 & 25 & & \it 8 & 27 & 28 & 31 & 32 \\ \hline

\it bin & \it Left & & & & & \it bin & \it Right & & & & \\ \hline

 & \it slot & \it 1,2 &\it 3,4 &\it 8,9 & \it 10,11 & & \it slot&\it 1,2 &\it 3,4 &
\it 8,9 &\it 10,11 \\ \hline

\it 1 & & 39,40 & 37,38 & 35,36 &34,33 &\it 4 & & 33,34 & 35,36 & 37,38 & 39,40 \\       

\it 2 & & 47,48 & 45,46 & 44,43 & 42,41 &\it 5 & & 41,42 & 43,44 & 45,46 & 47,48 \\ 

\it 3 & & 56,55 & 54,53 & 52,51 & 50,49 &\it 6 & & 49,50 & 51,52 & 53,54 & 55,56 \\
\hline \hline
\end{tabular}
\end{center}
\label{tab:firstdetail}
\end{table}
}

\begin{table}[!]
\caption[]{\it Correspondence between the 6 regions of the glass, and the
trigger units, 1 thru 4.}
\begin{center}
\begin{tabular}{|c|c|c|c|}
\hline
area & A+B & C+D & section \\ \hline
row & 1-31 & 1-31 & Protvino L and R\\
trigger & 1 & 2 & \\ \hline
row & 31-34 & 31-34 & boundary L and R\\
trigger & 1 & 2 & \\ \hline
row & 34-56 & 34-56 & RCS L and R \\
trigger & 3 & 4 & \\ \hline
\end{tabular}
\end{center}
\label{tab:triggerscheme}
\end{table}


The electronics located on the calorimeter platform include 224,   
{\bf first level} summing modules, each with 8 inputs, contained 
in 112 single width NIM modules, with analog output 
through 34-conductor flat cables in the back, amplified by a factor 
of 4.2, going to the 1792 ADC channels (48 are unused, and could 
become spares if needed). 
The first level summing modules also provide 4 amplified analog ``summed'' 
outputs with the same gain of 4.2, reducing the signals from each 
horizontal group of 8 PMs called A, B, C and D in Fig. \ref{fig:firstlevel}, 
to 1. Figure~\ref{fig:firstlevel} shows only the first two NIM bins for the 
lowest group of 
Protvino bars (rows 1,2,5,6) on the top left, then rows of 3,4,7,8 on the 
right. 
At the bottom the corresponding configuration for the RCS section is shown;
because each row in the RCS group is divided equally left and right the left 
NIM bin contains rows 33 to 40 left half, and the right NIM bin contains 
rows 33 to
40 right half. A complete scheme for all 1744 physical bars or 1792
electronic channels is found in Table \ref{tab:firstdetail}.
 
One of the four
parallel analog ``summed'' outputs goes to a discriminator located in the
same NIM bin, and then to a TDC, for a total of 224 TDCs. Another output 
(or two, as will be explained below) goes to the second level of summing 
modules where the ``summed'' output with gain of 1 are used, to further reduce the number
of analog channels by a factor of 8.
Both TDCs and ADCs 
are located on the separate electronic platform through 50 $\Omega$ cables, 
100 m long and 50 m long, respectively. 

\subsubsection{Second level summing}

The configuration of the {\bf second level summing modules} is explained in 
Fig. ~\ref{fig:secondlevel}. The
wiring is such that groups of 64 bars are added together, either $A+B$, or 
$C+D$,rows 1 
thru 4, and then the next group of 64 is made to overlap with the previous 
one by 1 row, as illustrated by the overlapping colors in Fig. 
\ref{fig:glassback}: the first group of 
64 bars in the lower left corner is reduced to 1 analog signal by the second 
level summing modules in the NIM bin located in the lower left hand 
corner of Fig. \ref{fig:secondlevel}: L, 1A,B thru 4A,B.
The overlap is obtained by repeating the first row (4A,B), in the next octet 
and then adding 
the next three rows: 4A,B thru 7A,B, an so on. All channels of the second 
level summing modules are shown in Fig. \ref{tab:firstdetail}.


\begin{figure}[h]
\begin{center}
\epsfig{file=/home/perdrisa/PS/calo_second_level_rev.eps,height=6.0in}
\caption[]{\it{Consists of 19 adders, with gain of 1; the front lemo 
provide analog AC and DC outputs. The wiring is such as to produce the 
overlap pattern shown in Fig. \ref{fig:glassback}. The concept involved is 
also illustrated in Fig. \ref{fig:idea}. Successive blocs of 64 bars left 
and rift are made to over lap by 1 row, left and right separately (hence 
no overlap between the two columns, A+B and C+D)}}.
\label{fig:secondlevel}
\end{center}
\end{figure}
Because of the vertical overlap scheme, a total of 38 summing octets (i.e. 19
NIM modules, each with 2 octets)
are required to achieve complete coverage. The second level adders send 
analog signals to the 4 trigger discriminator NIM units at the 
{\bf third level}. 

\begin{figure}[h]
\begin{center}
\epsfig{file=/home/perdrisa/PS/calo_third_level_rev.eps,height=6.0in}
\caption[]{\it{At third level of the trigger the 38 outputs form the 
second level ``adders'' are passed through discriminators with remotely 
adjustable threshold; these define the BC trigger level. The 38 logic 
outputs of the discriminators are then OR'ed to 1 in two steps, four fanin/Fanout's, followed by four logic OR's.. The wiring 
is such that the left- and right half of the calorimeter are grouped, 
allowing a different trigger level for the two vertical halves of BigCal.  }}
\label{fig:secondlevel}
\end{center}
\end{figure}

\subsubsection{Third level and trigger}

The schematics of the {\bf third level} is 
shown in Table \ref{tab:triggerscheme}. The 38 analog output signals from the second level summing
modules are sent to 4 discriminators modules numbered D1 thru D3, each with 16 inputs. D1 
receives the 11 sums of 64 for
rows 1 thru 34 A+B (i.e. left half), and D2 the 8 rows 34 thru 56 A+B. D3 receives the sums of 
64 for the 11
rows 1 thru 34 C+D (i.e. right half), and D4 the  8 rows 34 thru 56 C+D. The threshold of these 
discriminators is remotely adjustable, separately for each of the 4 modules (but not for each input).
One might choose a different threshold for the left and right halves of BigCal.
The logical output signal of the 38 discriminators circuits are then OR'd in 4 Fanin/Fanout 
NIM units, using 11 inputs in the first unit, 8 in the second, both for A+B. And similarly
11 inputs and 8 inputs for the third and fourth unit, both for C+D. The 4 outputs of the Fi/Fo 
are then OR'd in one logic AND unit in the mode
``any one of 4''.  Figure \ref{fig:thirdlevel} illustrated the above description.

\begin{figure}[h]
\begin{center}
\epsfig{file=/home/perdrisa/PS/calo_blockdiagram_full.eps,height=6.5in}
\caption[]{\it{Schematic representation of the electronics on the BigCal 
platform. }}
\label{fig:blockdiagram}
\end{center}
\end{figure}

\begin{figure}[h]
\begin{center}
\epsfig{file=/home/perdrisa/PS/calo_logic_idea_plus.eps,height=7.0in}
\caption[]{\it{Principle of the overlapping scheme at the second level: every 
third row has two first level output going into separate second level 
summing modules
represented by the black squares (the same black squares as in 
Fig. \ref{fig:glassback}, each representing an summing octet). }}
\label{fig:idea}
\end{center}
\end{figure}


\subsection{Important Procedures}

Always turn off BOTH: 1)HV supplies, and 2)Booster supplies BEFORE enetering the 
detector black box, or before working at the labirint openings, or before working 
on the monitoring system.

Avoid having booster supplies "ON" when the HV on any of the Protvino channel is "OFF".

Avoid having for a long time (days) the HV "ON" when the booster supply is "OFF" on 
the Protvino part.

Never use the monitoring system with more than 1KHz rate.

Be careful when connecting/disconnecting long signal RCS cables, never apply 
more than 2kg tension on the BNC connectors. The cylindrical HV connectors on 
the long HV cables are very fragile, always buble wrap them when disconnecting 
from the boxes and keep the connector in hand when running/laying the cable.

When transporting/moving the calorimeter tighten SLIGHTLY the bolts on the 
right (downstream) side of the glass frame. These bolts apply pressure on the 
glass. BE GENTLE, YOU CAN BREAK THE GLASS even using a small screw driver, instead 
of wrench. There are two bolts per 4 rows for the Protvino part (16 total) and 
two bolts per each row for the RCS part (48 total). Mount the 5 U-beams on the 
front calorimeter plane when transporting the detector.  

\subsection{Safety Issues}

The power supplies of the two FASTBUS crates at the electronics platform have currents up to 24A. 
Always turn the power down when working on the back side of the crates. Do not remove the safety 
covers on the back of the crates. If it is needed for the job, put them back right after the job is done.

Always turn off BOTH: 1)ALL HV supplies, and 2)ALL Booster supplies when working with any part 
of the HV circuit: HV crates and cards, short HV cables at the electronics side, HV boxes at the 
electronics side, long HV cables, HV boxes at the detector side, individual HV cables, bases, 
and PMTs. Always turn off BOTH: 1)ALL HV supplies, and the 2)ALL Booster supplies BEFORE 
enetering the detector black box. Each booster supply unit is capable of 1A current.

Be aware that in the Protvino part you may have voltage on channels even the HV is OFF for 
these channels, and the booster suplies are OFF. This is because the HV from other Protvino 
channels is connected to all the channels through the booster curcuit. The easiest way to be 
sure that there is no HV, is to turn OFF ALL the Protvino HV supplies.

When in the Hall, you are allowed to step on the calorimeter platform without harmed 
ONLY IF the (removable) fence all around the platform is in place.

The cooling system blows air inside the black box with a rate of 0.1m3 / sec which is a 
significant fire hazard. The interock system includes 4 thermosensors inside the black box, 
but it is supposed to turn off only the HV supplies. We do not intend to use the cooling 
system during the operation of the detector, since the tempreture increase when switching 
on the HVs ia about 100C. Therfore, if you have used the cooling system when working inside 
the detector, turn it OFF after the work.

Laser safety issues (expect Hamlet's contribution): we may actually use LED's. 


\end{document}